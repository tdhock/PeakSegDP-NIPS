\documentclass{article}
\usepackage{nips14submit_e,times}
\nipsfinalcopy
\usepackage{tikz}
\usepackage{verbatim}
\usepackage{hyperref} 
\usepackage{graphicx}
\usepackage{natbib}
\usepackage{amsmath,amssymb}
\DeclareMathOperator*{\argmin}{arg\,min}
\DeclareMathOperator*{\sign}{sign}
\DeclareMathOperator*{\Lik}{Lik}
\DeclareMathOperator*{\Peaks}{Peaks}
\DeclareMathOperator*{\HotSpots}{HotSpots}
\newcommand{\Cost}{\text{Cost}}
\DeclareMathOperator*{\Diag}{Diag}
\DeclareMathOperator*{\TPR}{TPR}
\DeclareMathOperator*{\Segments}{Segments}
\DeclareMathOperator*{\FPR}{FPR}
\DeclareMathOperator*{\argmax}{arg\,max}
\DeclareMathOperator*{\maximize}{maximize}
\DeclareMathOperator*{\minimize}{minimize}
\usepackage[margin=1in]{geometry}
\newcommand{\ZZ}{\mathbb Z}
\newcommand{\NN}{\mathbb N}
\newcommand{\RR}{\mathbb R}
\begin{document}

\title{PeakSeg: \textbf{Peak} detection via constrained optimal
  \textbf{Seg}mentation}

\author{Toby Dylan Hocking,
  Guillem Rigaill, and Guillaume Bourque}

\maketitle

\begin{abstract}
  Peak detection is a central problem in ChIP-seq data analysis, and
  current algorithms for this task are unsupervised and mostly
  effective for a single data type (e.g. histone H3K4me3 profiles with
  sharp peaks). We propose PeakSeg, a supervised peak detection
  algorithm based on constrained optimal segmentation, which is easy
  to tune since it has only one free parameter: the optimal number of
  peaks. We propose to tune it using annotated regions in a supervised
  penalty function learning problem, which we show results in
  state-of-the-art peak detection for both sharp H3K4me3 and broad
  H3K36me3 data types.
\end{abstract}

\section{Introduction to supervised ChIP-seq peak detection}

Chromatin immunoprecipitation sequencing (ChIP-seq) is a genome-wide
assay to profile histone modifications and transcription factor
binding sites, with many experimental and computational steps
\citep{practical}. Briefly, each experiment yields a set of sequence
reads which are aligned to a reference genome, and then the data are
interpreted by counting the number of aligned reads at each genomic
position. In this paper we propose a new method for peak calling these
data, which is a binary classification problem for each genomic
position. The positive class is enriched (peaks) and the negative
class is background noise.

More concretely, a ChIP-seq profile on a single
chromosome with $d$ base pairs is a vector $\mathbf y=
\left[
  \begin{array}{ccc}
    y_1 & \cdots & y_d
  \end{array}
\right]\in\ZZ_+^d$ of counts of aligned sequence reads. A peak
detection algorithm can be described as a function $c:\ZZ_+^d
\rightarrow \{0, 1\}^d$ which returns 0 for background noise and 1 for
a peak. In contrast to the supervised method proposed in this paper,
most previous algorithms are unsupervised since they define a peak
detector $c$ using only the profile data $\mathbf y$.

In supervised peak detection \citep{hocking2014visual}, there
are $n$ annotated samples, and each sample $i\in\{1, \dots, n\}$ has a
profile $\mathbf y_i\in\ZZ_+^d$ and a set of annotated regions $R_i$
which defines a non-convex annotation error function
\begin{equation}
  \label{eq:error}
  E[c(\mathbf y_i),  R_i] =
  \text{FP}[c(\mathbf y_i), R_i] +
  \text{FN}[c(\mathbf y_i), R_i].
\end{equation}
The annotation error counts the number of false positive (FP) and
false negative (FN) regions, so it takes values in the non-negative
integers. The goal is to find a peak caller with minimal error on some
test profiles:
\begin{equation}
  \label{eq:min_error}
  \minimize_c \sum_{i\in\text{test}} E[c(\mathbf y_i),  R_i].
\end{equation}

\section{Related work}

In the benchmark data set of \citet{hocking2014visual}, there are two
different histone mark types: H3K4me3 (sharp peaks) and H3K36me3
(broadly enriched regions). The best peak detection algorithm for
these H3K4me3 data was macs \citep{MACS}, and the best for H3K36me3
was HMCan \citep{HMCan}. Both of these algorithms are unsupervised,
but were calibrated using the annotated region labels to choose the
best scalar significance threshold hyperparameter via grid search.

The ChIP-seq segmentation model we propose is a constrained version of
the model proposed by \citet{cleynen2013segmentation}. Specifically,
they proposed to search all possible change-points to find the optimal
segmentation, but we propose to constrain the possible change-points to
the subset of models that can be interpreted as peaks.

\section{Unsupervised PeakSeg: segmenting one ChIP-seq profile}

\begin{figure}[b!]
  \centering
  \includegraphics[width=\textwidth]{figure-Segmentor-PeakSeg}
  %\input{figure-Segmentor-PeakSeg}
  \vskip -0.5cm
  \caption{Example profile $\mathbf y$ (black), with green horizontal
    lines for the segmentation mean $\mathbf m$, and green vertical
    lines to emphasize change-points. For this particular profile
    $\mathbf y$, the models are equivalent $\mathbf{\hat m}^s(\mathbf
    y) = \mathbf{\tilde m}^s(\mathbf y)$ for $s\in\{1, 3\}$ segments
    but not for $s\in\{5, 7\}$. For the constrained models, the 2nd,
    4th, ...  are interpreted as peaks (\ref{eq:peaks}), whose
    accuracy can be quantified using the annotations
    (\ref{eq:error}).}
  \label{fig:profiles}
\end{figure}

After fixing a maximum number of segments $1 \leq s_{\text{max}}\leq d$,
the unconstrained maximum likelihood segmentation problem is defined
for any $s\in\{1, 2, \dots, s_{\max}\}$ as
\begin{equation}
  \label{argmin:unconstrained}
  \begin{aligned}
    \mathbf{\hat m}^s(\mathbf y)  =\ 
    &\argmin_{\mathbf m\in\RR^{d}} && 
    \rho
    (\mathbf m, \mathbf y) \\
    \\
    &\text{such that} && \Segments(\mathbf m)=s,
  \end{aligned}
\end{equation}
where $\rho(\mathbf m, \mathbf y)= \sum_{j=1}^d m_j - y_j \log m_j$ is
the loss function corresponding to maximum likelihood inference of a
Poisson distribution with mean parameter $m_j$. The model complexity
$\Segments(\mathbf m)=1+\sum_{j=2}^d I(m_j \neq m_{j-1})$ is the
number of segments, where $I$ is the indicator function. Although it
is a non-convex optimization problem, the sequence of segmentations
$\mathbf{\hat m}^1(\mathbf y), \dots, \mathbf{\hat
  m}^{s_{\text{max}}}(\mathbf y)$ can be computed in $O(s_{\text{max}}
d^2)$ time using dynamic programming (DP) algorithms \citep{bellman},
or in $O(s_{\text{max}} d \log d)$ time using pruned DP
\citep{pruned-dp, Segmentor}.

We refer to (\ref{argmin:unconstrained}) as the unconstrained problem
since $\mathbf{\hat m}^s(\mathbf y)$ is the most likely segmentation
of all possible models with $s$ segments. Several unconstrained models
are shown on the left of Figure~\ref{fig:profiles}, and for example
the 2nd segment of the model with $s=3$ segments appears to capture
the peak in the data. 
% In general, we would like to use the 2nd, 4th,
% ... segments as peaks, and the 1st, 3rd, ... segments as
% background. 
To construct a peak detector $c$, first define the sign of the change
before base $j\in\{2, \dots, d\}$ as
\begin{equation}
  \label{eq:sign}
  S_j(\mathbf m) = \sign( m_{j} - m_{j-1} ),
\end{equation}
with $S_1(\mathbf m)=0$ by convention. Furthermore we define the number
of peaks at base $j\in\{1, \dots, d\}$ as
\begin{equation}
  \label{eq:peaks}
  P_j(\mathbf m) = \sum_{k=1}^j S_k(\mathbf m).
\end{equation}
In general for the unconstrained model $P_j(\mathbf m)\in\ZZ$, and for
example in Figure~\ref{fig:profiles} there is a position $j$ for which
$P_j\left[ \mathbf{\hat m}^5(\mathbf y) \right]=2$ (since the mean
changes up, up, down, down). We would like to constrain the number of
peaks $P_j(\mathbf m)\in\{0, 1\}$ so that we can use $c(\mathbf y) =
\mathbf P\left[ \mathbf{\tilde m}^s(\mathbf y) \right]$ as a peak
detector, where $\mathbf P[\mathbf m] = \left[\begin{array}{ccc}
    P_1(\mathbf m) & \cdots & P_d(\mathbf m)
\end{array}\right]\in\{0, 1\}^d$. That results 
in the constrained problem
\begin{equation}
  \label{argmin:constrained}
  \begin{aligned}
    \mathbf{\tilde m}^s(\mathbf y)  =\ 
    &\argmin_{\mathbf m\in\RR^{d}} && 
    %\sum_{j=1}^d \log\Lik
    \rho
    (\mathbf m, \mathbf y) \\
    \\
    &\text{such that} && \Segments(\mathbf m)=s,\\
    &&& P_j(\mathbf m) \in\{0, 1\} \text{ for all } j\in\{1, \dots, d\}.
  \end{aligned}
\end{equation}
Another way to interpret the constrained problem
(\ref{argmin:constrained}) is that the sequence of changes in the
segment means $\mathbf m$ must begin with a positive change and then
alternate: up, down, up, down, ... (and not up, up, down). Thus the
even numbered segments (2nd, 4th, etc) may be interpreted as peaks,
and the odd numbered segments (1st, 3rd, etc) may be interpreted as
background. Figure~\ref{fig:profiles} shows a profile where the
constraint is necessary to detect peaks for models with $s\in\{5, 7\}$
segments. We propose to use dynamic programming to compute the
sequence of maximum likelihood models $\mathbf{\tilde m}^1(\mathbf y),
\dots, \mathbf{\tilde m}^{s_{\text{max}}}(\mathbf y)$ satisfying this
up-down constraint.  The algorithm is in $O(s_{\text{max}} d^2)$ time,
where $d$ is the number of data points, using the compression scheme
proposed by \citet{Segmentor}.

\section{Supervised PeakSeg: learning a penalty function}
\label{sec:supervised}

After computing the constrained maximum likelihood segmentations for
each sample $i\in\{1,\dots, n\}$, the only question that remains is:
how many segments? To predict a sample-specific number of segments, we
propose to use the annotated regions to learn a penalty function
\citep{HOCKING-penalties}. Briefly, we define the optimal number of
segments 
\begin{equation}
  s^*(\lambda, \mathbf y) = 
  \argmin_{s\in\{1,3,\dots, s_{\text{max}}\}}
  \rho\left[
    \mathbf{\tilde m}^s(\mathbf y),
    \mathbf y
  \right]
  + \lambda s,
\end{equation}
$\lambda\in\RR_+$ is a positive
penalty value. We define sample-specific penalty values $\log
\lambda_i = f(\mathbf x_i) = \beta + \mathbf w^\intercal \mathbf x_i$,
which is an affine function with parameters $\beta\in\RR,\mathbf
w\in\RR^m$ that will be learned. We used an $m=2$-dimensional feature vector
$\mathbf x_i = \left[\begin{array}{cc} \log\max \mathbf y_i & \log d_i
\end{array}\right]$ where $d_i$ is the number of base pairs for 
sample/chromosome $i$. The learning algorithm amounts to minimizing a
smooth convex loss $\ell_i:\RR\rightarrow\RR_+$ which depends on the
annotated region data $R_i$:
\begin{equation}
  \label{eq:relax}
  \hat f = \argmin_f \sum_{i=1}^n
  \ell_i\left[ f(\mathbf x_i) \right].
\end{equation}
Since $\ell_i$ is smooth and convex, this problem can be easily solved
using gradient-based algorithms. We used the accelerated gradient
method of the FISTA algorithm \citep{fista}.

To make a prediction on a
test sample with profile $\mathbf y$ and features $\mathbf x$, we
compute the predicted penalty $\hat \lambda = \exp \hat f(\mathbf x)$,
the predicted number of segments $\hat s = s^*(\hat \lambda, \mathbf
y)$, and finally the predicted peaks $\mathbf P\left[ \mathbf{\tilde
    m}^{\hat s}(\mathbf y) \right]$.

\section{Results: state-of-the-art peak detection for two data types}

We downloaded 7 benchmark annotated region data
sets.\footnote{\url{http://cbio.ensmp.fr/~thocking/chip-seq-chunk-db/}}
Since the DP solver that we implemented has $O(s_{\text{max}} d^2)$
time complexity, we considered only the subset of samples which had at
most $d\leq 100,000$ data points to segment in each labeled genomic
window. This left a database of 4,628 annotated regions
% sum(unlist(sapply(dp.peaks.matrices, sapply, function(x)sum(x$regions))))
(out of 12,826 annotated regions in the entire benchmark). Finally,
since these genomic windows are relatively small, we set the maximum
number of segments $s_{\text{max}}=19$, and for each profile $\mathbf
y$ we computed $\mathbf{\tilde m}^1(\mathbf y), \dots, \mathbf{\tilde
  m}^{19}(\mathbf y)$ (\ref{argmin:constrained}). For the largest
profile we considered ($d=88,509$ data points), the DP algorithm
computed the 19 constrained optimal segmentations in 27 minutes.

We compared the proposed PeakSeg algorithm to the two previous
state-of-the-art peak detectors, macs and \mbox{hmcan.broad}. For each
data set, we randomly divided the annotated windows into half train
and half test. We trained the macs and hmcan.broad algorithms by
choosing the significance threshold with minimal annotation error on
the train set. We trained PeakSeg by learning a penalty function $\hat f$
on the train set (\ref{eq:relax}).

We show the percent test error for each algorithm and each data set in
Figure~\ref{fig:test-error}. As previously described
\citep{hocking2014visual}, macs had lower test error than
\mbox{hmcan.broad} for H3K4me3 data, and \mbox{hmcan.broad} had lower
test error than macs for H3K36me3 data. It is clear that our proposed
PeakSeg algorithm had lower test error than both macs and hmcan.broad
algorithms, for both data types.

% \begin{center}
%   \input{table-dp-peaks-regression}
% \end{center}

\section{Discussion, conclusions, and future work}

\begin{figure}[t!]
  \centering
  \includegraphics[width=\textwidth]{figure-dp-peaks-regression-dots}
  \vskip -0.5cm
  \caption{Test error of three algorithms on seven annotated region
    data sets (each point shows one of six randomly selected
    train/test splits). Data set names show histone mark type
    (e.g. H3K36me3), annotator (AM), and cell types (immune). PeakSeg
    had lower test error than macs and \mbox{hmcan.broad} for both
    H3K36me3 and H3K4me3 data.}
  \label{fig:test-error}
\end{figure}

We proposed to use the solution of a constrained optimal segmentation
problem (\ref{argmin:constrained}) as a ChIP-seq peak detector. For
$d$ data points and $s_{\text{max}}$ segments, we proposed to use
dynamic programming to compute the solutions in $O(s_{\text{max}} d^2)$
time. Furthermore, we proposed to use annotated region data as
supervision in a penalty learning problem (\ref{eq:relax}). This
approach yields state-of-the-art test error rates for peak detection
in a benchmark that includes both H3K4me3 and H3K36me3 data
sets.

PeakSeg is the first supervised peak detection algorithm. It has been
explicitly designed to take advantage of the annotated region data
proposed by \citet{hocking2014visual}. PeakSeg is also the first
algorithm to exhibit state-of-the-art accuracy on both sharp H3K4me3
and broad H3K36me3 ChIP-seq profiles. In the future, even better
accuracy may be obtained by engineering better features $\mathbf x_i$
for the penalty learning problem (\ref{eq:relax}), perhaps based on
Poisson segmentation model selection theory
\citep{cleynen2013segmentation}.

The current implementation of PeakSeg using dynamic programming has
one major limitation. The $O(s_{\text{max}} d^2)$ time complexity has
limited its application to subsets of chromosomes with $d\leq 100,000$
data points to segment. In comparison, the largest chromosome subset
in the benchmark has $d=254,451$ data points, and the largest
chromosome in hg19 (chr1) has $d=249,250,621$ base pairs. To apply
PeakSeg to these larger data sets, we are investigating a constrained
version of pruned dynamic programming \citep{pruned-dp, Segmentor},
which has $O(s_{\text{max}} d\log d)$ time complexity.

Finally, we are interested in segmenting multiple samples $i$ at the
same time, since peaks are often observed in the same genomic location
across several samples of the same cell type. This joint peak
detection problem may lead to more accurate peak calls, but it is a
considerably more difficult segmentation problem.

\bibliographystyle{abbrvnat}

\bibliography{refs}

\end{document}
